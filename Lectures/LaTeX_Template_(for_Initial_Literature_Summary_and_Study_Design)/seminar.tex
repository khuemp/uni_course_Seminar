\documentclass[	runningheads,
				a4paper]{llncs}

\usepackage{url}
\usepackage{graphicx}
\usepackage{amssymb}
\usepackage{hyperref}

% Support for special characters like "Umlaute"
\usepackage[utf8]{inputenc}


\usepackage[english]{babel}


%*********************************************************************%
% META                                                                %
%*********************************************************************%

\newcommand{\university}{Saarland University}
\newcommand{\school}{Saarland Informatics Campus}


\newcommand{\thetitle}{Seminar: Title of the Seminar}
\newcommand{\shorttitle}{Seminar: Title of the Seminar}
\newcommand{\thedate}{November 07}


\newcommand{\theforename}{Max}
\newcommand{\thesurname}{Mustermann}

% Advisors

\newcommand{\advisor}{Advisors}

\newcommand{\advisors}{Prof. Sven Apel, \\ \$ADVISOR}

% Title for the seminar

\newcommand{\theseminartitle}{Title of the paper}


%*********************************************************************%
% THE DOCUMENT                                                        %
%*********************************************************************%

\begin{document}
%*********************************************************************%
% TITLE                                                               %
%*********************************************************************%

% Arabic page numbering
\mainmatter 
	
% Title including a subtitle for the name of the seminar
\title{\theseminartitle \\ \small \thetitle}

% (Optional) In the case that the initial title is too long, the short title will be used

\author{}

% (Optional) This will appear near the page number
\authorrunning{\shorttitle}

\institute{}

\maketitle

%*********************************************************************%
% CONTENT                                                             %
%*********************************************************************%

% Introduction
\section{Introduction}

You are strongly encouraged to use \LaTeXe{} for the
preparation of your camera-ready manuscript together with the
corresponding Springer class file \verb+llncs.cls+. Only if you use
\LaTeXe{} can hyperlinks be generated in the online version
of your manuscript.

The \LaTeX{} source of this instruction file for \LaTeX{} users may be
used as a template. This is
located in the ``authors'' subdirectory in
\url{ftp://ftp.springer.de/pub/tex/latex/llncs/latex2e/instruct/} and
entitled \texttt{typeinst.tex}. There is a separate package for Word 
users. Kindly send the final and checked source
and PDF files of your paper to the Contact Volume Editor. This is
usually one of the organizers of the conference. You should make sure
that the \LaTeX{} and the PDF files are identical and correct and that
only one version of your paper is sent. It is not possible to update
files at a later stage. Please note that we do not need the printed
paper.

We would like to draw your attention to the fact that it is not possible
to modify a paper in any way, once it has been published. This applies
to both the printed book and the online version of the publication.
Every detail, including the order of the names of the authors, should
be checked before the paper is sent to the Volume Editors.

\subsection{Checking the PDF File}

Kindly assure that the Contact Volume Editor is given the name and email
address of the contact author for your paper. The Contact Volume Editor
uses these details to compile a list for our production department at
SPS in India. Once the files have been worked upon, SPS sends a copy of
the final pdf of each paper to its contact author. The contact author is
asked to check through the final pdf to make sure that no errors have
crept in during the transfer or preparation of the files. This should
not be seen as an opportunity to update or copyedit the papers, which is
not possible due to time constraints. Only errors introduced during the
preparation of the files will be corrected.

This round of checking takes place about two weeks after the files have
been sent to the Editorial by the Contact Volume Editor, i.e., roughly
seven weeks before the start of the conference for conference
proceedings, or seven weeks before the volume leaves the printer's, for
post-proceedings. If SPS does not receive a reply from a particular
contact author, within the timeframe given, then it is presumed that the
author has found no errors in the paper. The tight publication schedule
of LNCS does not allow SPS to send reminders or search for alternative
email addresses on the Internet.

In some cases, it is the Contact Volume Editor that checks all the final
pdfs. In such cases, the authors are not involved in the checking phase.

\subsection{Additional Information Required by the Volume Editor}

If you have more than one surname, please make sure that the Volume Editor
knows how you are to be listed in the author index.

\subsection{Copyright Forms}

The copyright form may be downloaded from the ``For Authors"
(Information for LNCS Authors) section of the LNCS Website:
\texttt{www.springer.com/lncs}. Please send your signed copyright form
to the Contact Volume Editor, either as a scanned pdf or by fax or by
courier. One author may sign on behalf of all of the other authors of a
particular paper. Digital signatures are acceptable.

\section{Paper Preparation}

Springer provides you with a complete integrated \LaTeX{} document class
(\texttt{llncs.cls}) for multi-author books such as those in the LNCS
series. Papers not complying with the LNCS style will be reformatted.
This can lead to an increase in the overall number of pages. We would
therefore urge you not to squash your paper.

Please always cancel any superfluous definitions that are
not actually used in your text. If you do not, these may conflict with
the definitions of the macro package, causing changes in the structure
of the text and leading to numerous mistakes in the proofs.

If you wonder what \LaTeX{} is and where it can be obtained, see the
``\textit{LaTeX project site}'' (\url{http://www.latex-project.org})
and especially the webpage ``\textit{How to get it}''
(\url{http://www.latex-project.org/ftp.html}) respectively.

When you use \LaTeX\ together with our document class file,
\texttt{llncs.cls},
your text is typeset automatically in Computer Modern Roman (CM) fonts.
Please do
\emph{not} change the preset fonts. If you have to use fonts other
than the preset fonts, kindly submit these with your files.

Please use the commands \verb+\label+ and \verb+\ref+ for
cross-references and the commands \verb+\bibitem+ and \verb+\cite+ for
references to the bibliography, to enable us to create hyperlinks at
these places.

For preparing your figures electronically and integrating them into
your source file we recommend using the standard \LaTeX{} \verb+graphics+ or
\verb+graphicx+ package. These provide the \verb+\includegraphics+ command.
In general, please refrain from using the \verb+\special+ command.

Remember to submit any further style files and
fonts you have used together with your source files.

\subsubsection{Headings.}

Headings should be capitalized
(i.e., nouns, verbs, and all other words
except articles, prepositions, and conjunctions should be set with an
initial capital) and should,
with the exception of the title, be aligned to the left.
Words joined by a hyphen are subject to a special rule. If the first
word can stand alone, the second word should be capitalized.

Here are some examples of headings: ``Criteria to Disprove
Context-Freeness of Collage Language", ``On Correcting the Intrusion of
Tracing Non-deterministic Programs by Software", ``A User-Friendly and
Extendable Data Distribution System", ``Multi-flip Networks:
Parallelizing GenSAT", ``Self-determinations of Man".

\subsubsection{Lemmas, Propositions, and Theorems.}

The numbers accorded to lemmas, propositions, and theorems, etc. should
appear in consecutive order, starting with Lemma 1, and not, for
example, with Lemma 11.

	
\section{Conclusion}

%*********************************************************************%
% APPENDIX                                                            %
%*********************************************************************%

\appendix
\section{Appendix}
% Insert the appendix here. You can alternatively include files via: \include{pathToFile}

%*********************************************************************%
% LITERATURE                                                          %
%*********************************************************************%
% As a recommendation JabRef might be a usefull tool for this section. Use myRefs.bib therefore
\phantomsection
\bibliographystyle{splncs03}
\bibliography{literature}
	
\end{document}
